\end{multicols}
\begin{center}
\begin{longtable}{llcl}

\caption{Species list, trap type species collected with, and crop species collected off of, for a collection of spiders from agricultural fields in the Matanuska-Susitna Valley, Alaska.\label{spider_species}}\\

\hline
\\[-1.0em]
\hline
\\[-1.0em]
\bf{Family}	&	\bf{Species}	&	\bf{Trap type}	&	\bf{Crop}	\\
\hline
\\[-1.0em]
\endfirsthead

\hline
\bf{Family}	&	\bf{Species}	&	\bf{Trap type}	&	\bf{Crop}	\\
\hline
\endhead

\multicolumn{4}{c}{\textit{Continued on next page\ldots{}}}\\
\hline
\endfoot

\hline
\\[-1.0em]
\hline
\endlastfoot

Aranaeidae	& \textit{Araniella proxima} (Kulczy\'{n}ski, 1885)	&	\acr{APT}	&	rhubarb	\\
	& \textit{Larinioides cornutus} (Clerck, 1757)	&	\acr{APT}	&	rhubarb	\\
	& \textit{Larinioides patagiatus} (Clerck, 1757)	&	\acr{APT}, \acr{CBT}	&	rhubarb	\\
Clubionidae	& \textit{Clubiona furcata} Emerton, 1919	&	\acr{CBT}	&	rhubarb	\\
	& \textit{Clubiona riparia} L. Koch, 1866	&	\acr{APT}, \acr{CBT}	&	rhubarb	\\
Dictynidae	& \textit{Dictyna brevitarsa} Emerton, 1915	&	\acr{APT}	&	rhubarb	\\
	& \textit{Dictyna maj}or Menge, 1869 	&	\acr{APT}, \acr{CBT}	&	rhubarb	\\
	& \textit{Emblyna annulipes} (Blackwall, 1846)	&	\acr{APT}, \acr{CBT}	&	lettuce, rhubarb	\\
	& \textit{Emblyna manitoba} (Ivie, 1947)	&	\acr{APT}	&	rhubarb	\\
Gnaphosidae	& \textit{Zelotes fratris} Chamberlin, 1920	&	\acr{CBT}	&		\\
Linyphiidae	& \textit{Agyneta lophophor} (Chamberlin \& Ivie, 1933)	&	\acr{CBT}	&		\\
	& \textit{Allomengea dentisetis} (Grube, 1861)	&	\acr{CBT}	&	rhubarb	\\
	& \textit{Bathyphantes brevipes} (Emerton, 1917)	&	\acr{APT}	&	rhubarb	\\
	& \textit{Bathyphantes latescens} (Chamberlin, 1919)	&	\acr{APT}, \acr{CBT}	&	rhubarb	\\
	& \textit{Bathyphantes pallidus} (Banks, 1892) 	&	\acr{CBT}	&	rhubarb	\\
	& \textit{Centromerus sylvaticus} (Blackwall, 1841)	&	\acr{APT}, \acr{CBT}	&	rhubarb	\\
	& \textit{Diplocephalus subrostratus} (O. Pickard-Cambridge, 1873)	&		&	potato	\\
	& \textit{Erigone arctica} (White, 1852) 	&	\acr{CBT}	&	lettuce	\\
	& \textit{Erigone atra} Blackwall, 1833	&	\acr{APT}, \acr{CBT}	&	lettuce, rhubarb	\\
	& \textit{Erigone blaesa} Crosby \& Bishop, 1928	&	\acr{CBT}	&	lettuce, potato	\\
	& \textit{Erigone dentigera} O. Pickard-Cambridge, 1874	&	\acr{APT}	&	rhubarb	\\
	& \textit{Erigone tanana} Chamberlin \& Ivie, 1947	&	\acr{APT}	&		\\
	& \textit{Gnathonarium taczanowskii} (O. Pickard-Cambridge, 1873)	&	\acr{APT}, \acr{CBT}	&	lettuce, rhubarb	\\
	& \textit{Mecynargus paetulus} (O. Pickard-Cambridge, 1875)	&	\acr{APT}, \acr{CBT}	&	rhubarb	\\
	& \textit{Microlinyphia pusilla} (Sundevall, 1830)	&	\acr{APT}, \acr{CBT}	&	rhubarb	\\
	& \textit{Microneta viaria} (Blackwall, 1841)	&	\acr{CBT}	&	lettuce, rhubarb	\\
	& Misc \#3	&	\acr{APT}, \acr{CBT}	&	rhubarb	\\
	& \textit{Phlattothrata parva} (Kulczy\'{n}ski, 1926)	&	\acr{APT}, \acr{CBT}	&	rhubarb	\\
	& \textit{Praestigia kulczynskii} Eskov, 1979	&	\acr{CBT}	&	rhubarb	\\
	& \textit{Walckenaeria atrotibialis} (O. Pickard-Cambridge, 1878) 	&	\acr{CBT}	&	rhubarb	\\
Lycosidae	& \textit{Pardosa palustris} (Linnaeus, 1758) 	&	\acr{CBT}	&	rhubarb	\\
Philodromidae	& \textit{Rhysodromus alascensis} (Keyserling, 1884)	&	\acr{CBT}	&	rhubarb	\\
	& \textit{Philodromus cespitum} (Walckenaer, 1802)	&	\acr{APT}, \acr{CBT}	&	rhubarb	\\
	& \textit{Philodromus placidus} Banks, 1892 	&	\acr{APT}	&	potato	\\
	& \textit{Thanatus striatus} C. L. Koch, 1845 	&	\acr{CBT}	&	rhubarb	\\
Salticidae	& \textit{Attulus striatus} (Emerton, 1911)	&	\acr{APT}	&	rhubarb	\\
	& \textit{Pelegrina montana} (Emerton, 1891)	&	\acr{APT}	&	rhubarb	\\
Tetragnathidae	& \textit{Tetragnatha dearmata} Thorell, 1873	&	\acr{APT}	&	rhubarb	\\
	& \textit{Tetragnatha laboriosa} Hentz, 1850	&	\acr{APT}	&	rhubarb	\\
Thomsidae	& \textit{Misumena vatia} (Clerck, 1757)	&	\acr{APT}	&	rhubarb	\\
	& \textit{Ozyptila gertschi} Kurata, 1944	&	\acr{CBT}	&	rhubarb	\\
	& \textit{Ozyptila sincera canadensis} Dondale \& Redner, 1975	&	\acr{CBT}	&	rhubarb	\\


\end{longtable}
%\end{center}
%\begin{multicols}{2}
