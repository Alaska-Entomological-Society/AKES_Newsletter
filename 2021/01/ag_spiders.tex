\title{Notes on a collection of spiders from agricultural by-catch from the Matanuska-Susitna area of Alaska}

\author{by Jozef Slowik\footnote{University of Alaska Cooperative Extension Service}}

\maketitle

Spiders are in important natural predator in many ecosystems, though their contribution to pest control in agroecosystems does seem unclear \citep{NyffelerBirkhofer2017}. This is largely due to the limited habitat and spider diversity found in agricultural plots \citep{NyffelerSunderland2003}. However, spiders are often specialists and recognizing which species may assist with a pest problem, and enhancing the field to encourage that species may be beneficial \citep{Rypstraetal1999}. There have been no studies on arachnids from agriculture in Alaska. This study looked at by-catch from several other pest related studies around Palmer, Alaska and identified the species associated with those agriculture systems.

\section{Methods}

Spider specimens were removed from by-catch vials collected as part of several \acr{USDA} projects examining pests of crops grown in the Matanuska-Susitna valley. Fields sampled were at the Matanuska Experiment Farm and Extension Center (N 61.5686\textdegree{} W 149.2495\textdegree{}), The Point Mackenzie Correctional Farm (N 61.4186\textdegree{} W 150.0924\textdegree{}). Additionally, fields at Pyrah’s (N 61.5319\textdegree{} W 149.0823\textdegree{}), Cambell’s (N 61.5153\textdegree{} W 149.0800\textdegree{}), and VanderWheel’s (N 61.5634\textdegree{} W 149.1495\textdegree{}) commercial farms were sampled. 

Specimens were collected using either Aphid pan traps (\acr{APT}) \citep{Pantojaetal2010a} or click beetle traps (\acr{CBT}) \citep{Pantojaetal2010b}. Specimens were collected off rhubarb (\textit{Rheum} spp.), potato (\textit{Solanum tuberosum} (L.)), or lettuce (\textit{Lactuca sativa} L.). Collections occurred over the years 2006--2008. 

A total of 1812 by-catch vials were examined. Spiders were identified to species if mature, and family if juvenile using \citet{Ubicketal2005}. 
Family guild organization follows \citet{Uetzetal1999}. Not all by-catch samples included crop information (24\%). All specimens have been deposited at the University of Alaska Museum of the North.

\section{Results}
Of the 1812 by-catch vials examined 165 (9.1\%) contained spiders. A total of 262 spiders were found, of those 176 were adult and could be identified to species. Forty-two species were identified representing 10 families (Table \ref{spider_species}). One additional family, Theridiidae, was only represented by juveniles. The highest number of specimens and species was exhibited by the family Linyphiidae (Table \ref{spider_families}). Rhubarb samples dominated the crop type sampled ($n=133$) compared to potato ($n=10$) and lettuce ($n=7$). The Wandering Sheet guild dominated species and samples because of the inclusion of Linyphiidae. Second was the Ambusher (Thomisidae and Philodromidae) and Orbweaver (Araneidae and Tetragnathidae) guilds (Table 3). 

\end{multicols}
\begin{center}
\begin{longtable}{llcl}

\caption{Species list, trap type species collected with, and crop species collected off of, for a collection of spiders from agricultural fields in the Matanuska-Susitna Valley, Alaska.\label{spider_species}}\\

\hline
\\[-1.0em]
\hline
\\[-1.0em]
\bf{Family}	&	\bf{Species}	&	\bf{Trap type}	&	\bf{Crop}	\\
\hline
\\[-1.0em]
\endfirsthead

\hline
\bf{Family}	&	\bf{Species}	&	\bf{Trap type}	&	\bf{Crop}	\\
\hline
\endhead

\multicolumn{4}{c}{\textit{Continued on next page\ldots{}}}\\
\hline
\endfoot

\hline
\\[-1.0em]
\hline
\endlastfoot

Aranaeidae	& \textit{Araniella proxima} (Kulczy\'{n}ski, 1885)	&	\acr{APT}	&	rhubarb	\\
	& \textit{Larinioides cornutus} (Clerck, 1757)	&	\acr{APT}	&	rhubarb	\\
	& \textit{Larinioides patagiatus} (Clerck, 1757)	&	\acr{APT}, \acr{CBT}	&	rhubarb	\\
Clubionidae	& \textit{Clubiona furcata} Emerton, 1919	&	\acr{CBT}	&	rhubarb	\\
	& \textit{Clubiona riparia} L. Koch, 1866	&	\acr{APT}, \acr{CBT}	&	rhubarb	\\
Dictynidae	& \textit{Dictyna brevitarsa} Emerton, 1915	&	\acr{APT}	&	rhubarb	\\
	& \textit{Dictyna maj}or Menge, 1869 	&	\acr{APT}, \acr{CBT}	&	rhubarb	\\
	& \textit{Emblyna annulipes} (Blackwall, 1846)	&	\acr{APT}, \acr{CBT}	&	lettuce, rhubarb	\\
	& \textit{Emblyna manitoba} (Ivie, 1947)	&	\acr{APT}	&	rhubarb	\\
Gnaphosidae	& \textit{Zelotes fratris} Chamberlin, 1920	&	\acr{CBT}	&		\\
Linyphiidae	& \textit{Agyneta lophophor} (Chamberlin \& Ivie, 1933)	&	\acr{CBT}	&		\\
	& \textit{Allomengea dentisetis} (Grube, 1861)	&	\acr{CBT}	&	rhubarb	\\
	& \textit{Bathyphantes brevipes} (Emerton, 1917)	&	\acr{APT}	&	rhubarb	\\
	& \textit{Bathyphantes latescens} (Chamberlin, 1919)	&	\acr{APT}, \acr{CBT}	&	rhubarb	\\
	& \textit{Bathyphantes pallidus} (Banks, 1892) 	&	\acr{CBT}	&	rhubarb	\\
	& \textit{Centromerus sylvaticus} (Blackwall, 1841)	&	\acr{APT}, \acr{CBT}	&	rhubarb	\\
	& \textit{Diplocephalus subrostratus} (O. Pickard-Cambridge, 1873)	&		&	potato	\\
	& \textit{Erigone arctica} (White, 1852) 	&	\acr{CBT}	&	lettuce	\\
	& \textit{Erigone atra} Blackwall, 1833	&	\acr{APT}, \acr{CBT}	&	lettuce, rhubarb	\\
	& \textit{Erigone blaesa} Crosby \& Bishop, 1928	&	\acr{CBT}	&	lettuce, potato	\\
	& \textit{Erigone dentigera} O. Pickard-Cambridge, 1874	&	\acr{APT}	&	rhubarb	\\
	& \textit{Erigone tanana} Chamberlin \& Ivie, 1947	&	\acr{APT}	&		\\
	& \textit{Gnathonarium taczanowskii} (O. Pickard-Cambridge, 1873)	&	\acr{APT}, \acr{CBT}	&	lettuce, rhubarb	\\
	& \textit{Mecynargus paetulus} (O. Pickard-Cambridge, 1875)	&	\acr{APT}, \acr{CBT}	&	rhubarb	\\
	& \textit{Microlinyphia pusilla} (Sundevall, 1830)	&	\acr{APT}, \acr{CBT}	&	rhubarb	\\
	& \textit{Microneta viaria} (Blackwall, 1841)	&	\acr{CBT}	&	lettuce, rhubarb	\\
	& Misc \#3	&	\acr{APT}, \acr{CBT}	&	rhubarb	\\
	& \textit{Phlattothrata parva} (Kulczy\'{n}ski, 1926)	&	\acr{APT}, \acr{CBT}	&	rhubarb	\\
	& \textit{Praestigia kulczynskii} Eskov, 1979	&	\acr{CBT}	&	rhubarb	\\
	& \textit{Walckenaeria atrotibialis} (O. Pickard-Cambridge, 1878) 	&	\acr{CBT}	&	rhubarb	\\
Lycosidae	& \textit{Pardosa palustris} (Linnaeus, 1758) 	&	\acr{CBT}	&	rhubarb	\\
Philodromidae	& \textit{Rhysodromus alascensis} (Keyserling, 1884)	&	\acr{CBT}	&	rhubarb	\\
	& \textit{Philodromus cespitum} (Walckenaer, 1802)	&	\acr{APT}, \acr{CBT}	&	rhubarb	\\
	& \textit{Philodromus placidus} Banks, 1892 	&	\acr{APT}	&	potato	\\
	& \textit{Thanatus striatus} C. L. Koch, 1845 	&	\acr{CBT}	&	rhubarb	\\
Salticidae	& \textit{Attulus striatus} (Emerton, 1911)	&	\acr{APT}	&	rhubarb	\\
	& \textit{Pelegrina montana} (Emerton, 1891)	&	\acr{APT}	&	rhubarb	\\
Tetragnathidae	& \textit{Tetragnatha dearmata} Thorell, 1873	&	\acr{APT}	&	rhubarb	\\
	& \textit{Tetragnatha laboriosa} Hentz, 1850	&	\acr{APT}	&	rhubarb	\\
Thomsidae	& \textit{Misumena vatia} (Clerck, 1757)	&	\acr{APT}	&	rhubarb	\\
	& \textit{Ozyptila gertschi} Kurata, 1944	&	\acr{CBT}	&	rhubarb	\\
	& \textit{Ozyptila sincera canadensis} Dondale \& Redner, 1975	&	\acr{CBT}	&	rhubarb	\\


\end{longtable}
%\end{center}
%\begin{multicols}{2}


%\end{multicols}
%\begin{center}
\begin{longtable}{lrrrr}

\caption{Spider families, number of specimens, and number of representative species for a collection of spiders from agricultural fields in the Matanuska-Susitna Valley, Alaska.\label{spider_families}}\\

\hline
\\[-1.0em]
\hline
\\[-1.0em]
\bf{Family}	&	\bf{Number of specimens}	&	\bf{Percent of specimens}	&	\bf{Number of species}	&	\bf{Percent of species}	\\
\hline
\\[-1.0em]
\endfirsthead

\hline
\bf{Family}	&	\bf{Species}	&	\bf{Trap type}	&	\bf{Crop}	\\
\hline
\endhead

\multicolumn{5}{c}{\textit{Continued on next page\ldots{}}}\\
\hline
\endfoot

\hline
\\[-1.0em]
\hline
\endlastfoot


Araneidae	&	16	&	6.1	&	3	&	7.1	\\
Clubionidae	&	9	&	3.4	&	2	&	4.8	\\
Dictynidae	&	11	&	4.2	&	4	&	9.5	\\
Gnaphosidae	&	1	&	0.4	&	1	&	2.4	\\
Linyphiidae	&	169	&	64.5	&	20	&	47.6	\\
Lycosidae	&	5	&	1.9	&	1	&	2.4	\\
Philodromidae	&	14	&	5.3	&	4	&	9.5	\\
Salticidae	&	4	&	1.5	&	2	&	4.8	\\
Tetragnathidae	&	13	&	5.0	&	2	&	4.8	\\
Theridiidae	&	5	&	1.9	&	0	&	0	\\
Thomisidae	&	15	&	5.7	&	3	&	7.1	\\


\end{longtable}
%\end{center}
%\begin{multicols}{2}


\section{Discussion}

These data represent the first sampling of spider species occurring in association with agriculture in Alaska. It is also the first data on species occurrences in Southcentral Alaska since \citeauthor{ChamberlinIvie1947}’s work in \citeyear{ChamberlinIvie1947}. Although these data are from agricultural settings, the data are representative of other diversity surveys in Alaska \citep{Slowik2006, SlowikBlagoev2012, Sikesetal2013}.  These data do differ from agricultural data from the contiguous United States in which Linyphiidae do make up the most diverse group found, but they are not the most commonly collected family \citep{YoungEdwards1990}, but they agree with agricultural data from Europe \citep{NyffelerBirkhofer2017}. 

Generally wandering guilds including the families Salticidae and Lycosidae will make up a larger proportion of specimens and species \citep{YoungEdwards1990, Kerzicniketal2013}. However, this is likely an artifact of the collection methods as both traps are above ground level requiring the spider to climb up the vegetation to become victim of the trap. Given that only 9.1\% of samples had spiders in the samples, this implies that the trapping methods were not very efficient at trapping spiders. For an \acr{IPM} application these traps were effective at collecting their intended pest and spared natural predators like spiders, though the true abundance of spiders is not known for comparison. Unfortunately, crop comparison could not be conducted due to the low number of samples from potato and lettuce, but these data are an initial baseline to build on. 

\section{Acknowledgements}

Thanks to Jodie Anderson at \acr{MEFEC} for providing samples and holding on to them. Additional thanks to Alberto Pantoja of the \acr{USDA} and J.\ Malipanis and S.\ Lillard for collection and sorting of specimens. 

\bibliography{ag_spiders}
