\title{Miller Creek Watershed aquatic invertebrate inventory}

\author{by
 Matthew L.\ Bowser\footnoteremember{KenaiNWR}{USFWS \href{https://www.fws.gov/refuge/kenai/}{Kenai National Wildlife Refuge}, Soldotna, Alaska}\footnote{\email{matt\_bowser@fws.gov}},
 Samuel I.\ Artaiz\footnoterecall{KenaiNWR},
 Jake M.\ Danner\footnoterecall{KenaiNWR},
 Kris K.\ Dent\footnoteremember{SoldotnaADFG}{\href{https://www.adfg.alaska.gov/}{Alaska Department of Fish and Game}, Soldotna, Alaska},
 Robert L.\ Massengill\footnoterecall{SoldotnaADFG},
 Benjamin Meyer\footnoteremember{KWF}{\href{https://kenaiwatershed.org/}{Kenai Watershed Forum}, Soldotna, Alaska},
 Dom Watts\footnoterecall{KenaiNWR}, 
 Chelsea Wisotzkey\footnoterecall{KWF}, and
 Warren R.\ Wyrick\footnoterecall{SoldotnaADFG}
 
 \parbox[t][][s]{\columnwidth}{\vspace{3em}} % This box exists just to prevent the authors from breaking over to a new column.
 }

\maketitle

\end{multicols}

\vspace{-1cm}
\begin{center}
 \parbox[t][][s]{14cm}{\section{Abstract}
Because benthic macroinvertebrates and zooplankton are susceptible to rotenone, surveys of freshwater macroinvertebrates were conducted in the Miller Creek Watershed, Kenai Peninsula, Alaska ahead of a planned treatment using the pesticide rotenone in fall 2021. Currently, 32 of 32 planned samples have been collected in 2021. Another 32 post-treatment invertebrate samples are planned in 2022 to enable comparison of pre- and post-treatment freshwater invertebrate communities.}
 
\end{center}

\vspace{4mm}

\begin{multicols}{2}

\section{Introduction}

Invasive northern pike (\textit{Esox lucius} Linnaeus, 1758) were discovered in the Miller Creek Watershed in 2019 \citep{KNWR2021} and a rotenone treatment to eradicate pike from the drainage is planned for fall 2021.

Benthic macroinvertebrates and zooplankton are susceptible to rotenone \citep{Finlaysonetal2018}, so \citet{Finlaysonetal2018} recommended sampling aquatic invertebrates before and after rotenone treatments to document survival and recovery of these animals.

% Quote below from Massengill (2014).
%In Southcentral Alaska, the effect of rotenone to aquatic invertebrate communities is typically temporary in nature and usually requires 1–3 years for posttreatment levels of zooplankton to be restored to pretreatment levels (Chlupach 1977). This is longer than reported in many other areas of North America where invertebrate recovery often takes a year or less (Kiser et al. 1963; Hamilton et al. 2009). Other studies show that zooplankton such as cladocerans and copepods have rotenone resistant eggs capable of reseeding a lake after a rotenone treatment (Bradbury 1986; Melaas et al. 2001). Fall applications may help zooplankton communities recover because many species are in rotenone-resistant life stages and there is time for population recovery before spring (Melaas et al. 2001).

\section{Methods}

We generally followed the methods used in previous aquatic inventories that were performed before and after rotenone applications on the Kenai Peninsula \citep{Massengill2014, Massengill2017} with the exception that identifications are to be obtained by metabarcoding instead of morphological identifications.

Aquatic mollusks were sampled under Alaska Department and Fish and Game Aquatic Resource Permit No.\ SF2021-134.

\subsection{Study area}

The waterbodies surveyed were those in which rotenone will be applied, the same water bodies in which northern pike have been detected. These include North Vogel Lake, Vogel Lake, and Miller Creek. North Vogel Lake (14~ha) and Vogel Lake (49~ha) are separated by about 200~m and connected by a small stream that flows through a wetland. Both lakes are similar in terms of water quality metrics \citep{Meyer2021}. Miller Creek drains Vogel Lake, running approximately 7~km to Cook Inlet.

\subsection{Field sampling}

Each of the three waterbodies is to be surveyed once in July and once in August to capture some of the seasonal phenology of invertebrates. Sampling will be conducted in the same temporal windows in 2021 and 2022 after a potential rotenone application in fall 2021.

In the two lakes, sampling methods will include plankton samples using a Wisconsin net, littoral sampling using D-nets, and benthic samples using an Ekman dredge, generally following the methods of \citet{Massengill2014, Massengill2017} for lake sampling. Only D-nets will be used in Miller Creek.

Sample sizes were chosen to be similar to those used by \citet{Massengill2014, Massengill2017}. \citet{Massengill2014}, to survey invertebrates in 38~ha Scout Lake, collected 3 light trap samples, 5 Ekman samples, 5 D-net samples, and 2 Wisconsin net samples. For 157~ha Stormy Lake, \citet{Massengill2014, Massengill2017} collected 5 Ekman samples, 5 D-net samples, and 3 Wisconsin net sample.

At each of two visits per year we proposed to collect 3 D-net samples, 3 Ekman dredge samples, and 2 Wisconsin net samples in Vogel Lake; 2 D-net samples, 2 Ekman dredge samples, and 1 Wisconsin net sample in North Vogel Lake; and 3 D-net samples in Miller Creek, a total of 16 invertebrate samples per visit (Table \ref{tab:nsamples}), 32 samples per year, and 64 samples over the two year project. 


At North Vogel Lake, Vogel Lake, and upper Miller Creek we sampled twice in 2021: first on July 20--23 and second on August 28. We sampled at 15 sites using three methods. Samples near the mouth of Miller Creek were collected on September 13, 2021. In addition, invertebrates were collected opportunistically while the authors were working within the study area. 

\end{multicols}

\begin{table}[h]
	\caption{Numbers of samples to be collected in the three water bodies by sampling method.}
	\centering
	\begin{tabular}{lrrr|r}
	\hline
		\hline
	
		Water body & D-net & Ekman dredge & Wisconsin net & Total \\
		\hline
    North Vogel Lake & 2 & 2 & 1 &  5 \\
    Vogel Lake       & 3 & 3 & 2 &  8 \\
    Miller Creek     & 3 & 0 & 0 &  3 \\
   \hline
Total            & 8 & 5 & 3 & 16 \\
		\hline
		\hline
	\end{tabular}
	\label{tab:nsamples}
\end{table}
\begin{multicols}{2}

Samples were collected into ethanol or SK picglobal 99.9\% pure propylene glycol. Samples intended for metabarcoding were collected into propylene glycol or, in the case of two samples, collected into ethanol and then transferred to propylene glycol.

\subsection{Specimen processing}

Specimens were deposited in the Kenai National Wildlife Refuge's entomology and invertebrate collections (international collection coden: \acr{KNWR}) and managed with Arctos (\url{https://arctos.database.museum/}).



For the morphologically identified specimens, we used the following references: \cite{Borroretal1989}, \cite{Collet2008}, \cite{Durfee2005}, \cite{Hatch1953}, \cite{Kenner2009}, \cite{MerrittCummins1996}, \cite{Merrittetal2008}, \cite{Reid1987}, \cite{Roughley2000}, \cite{Wallis1933}.

Metabarcoding samples were stored in a -23~\textdegree{}C freezer exept when samples were being processed. Invertebrates were separated from debris by hand under a dissecting microscope. Care was taken to reduce possible cross-contamination of \acr{DNA} among samples. Samples were shipped out on ice on September 29, 2021, arriving the next day at Molecular Research Laboratory (Shallowater, Texas, \url{http://www.mrdnalab.com}).

We chose to use the \textit{mlCOIintF}/\textit{jgHCO2198} (\acr{G\-G\-W\-A\-C\-W\-G\-G\-W\-T\-G\-A\-A\-C\-W\-G\-T\-W\-T\-A\-Y\-C\-C\-Y\-C\-C}\-/\-\acr{T\-A\-I\-A\-C\-Y\-T\-C\-I\-G\-G\-R\-T\-G\-I\-C\-C\-R\-A\-A\-R\-A\-A\-Y\-C\-A}) primer set of \citet{Lerayetal2013} for \acr{PCR}, targeting a 313~bp region of the \acr{COI} \acr{DNA} barcoding region. This primer set has been successfully used for a wide variety of invertebrates, including terrestrial invertebrates \citep{Bowseretal2020}, freshwater macroinvertebrates \citep{Hajibabaeietal2019}, freshwater plankton \citep{Yangetal2017}, and stomach contents of freshwater fish \citep{BowserBowser2020}.

\subsection{Analysis}

We plan to process raw read data using MetaWorks \citep{PorterHajibabaei2020}, running this on the Yeti supercomputer \citep{USGSARC2021}. As in \citet{Massengill2014, Massengill2017}, the metric will be simple presence or absence of taxa within the lakes before and after treatment.


\section{Results}

Data and images are available via an Arctos project (\url{https://arctos.database.museum/project/10003613}). Specimen data are available via a search for records from this project (\url{https://arctos.database.museum/SpecimenResults.cfm?project_id=10003613}). All Arctos records from this project are also published to the Global Biodiversity Information Facility (\acr{GBIF}, \url{https://www.gbif.org/}) via the VertNet Integrated Publishing Toolkit (\url{http://ipt.vertnet.org/}). ``Research grade'' iNaturalist observations are also published to \acr{GBIF}.

Georeferenced photo observations have been gathered in a project on iNaturalist.org at \url{https://www.inaturalist.org/projects/miller-creek-invertebrate-inventory}. 

\section{Discussion}

\section{Acknowledgments}

We thank \href{https://www.inaturalist.org/}{iNaturalist.org} people \verb|amr_mn| and \verb|zvkemp| (Zach Kemp) for identifications.

\bibliography{Miller_Creek_inventory}


