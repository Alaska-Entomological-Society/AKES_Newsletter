\title{Update to the identification guide to female Alaskan bumble bees and a summary of recent changes to the Alaskan bumble bee fauna}

\subtitle{\doi{10.----------}}

\author{by 
 Derek S.\ Sikes\footnote{University of Alaska Museum, Institute of Arctic Biology, University of Alaska Fairbanks, Fairbanks, Alaska, USA \email{dssikes@alaska.edu}} and
 Jessica J.\ Rykken\footnote{Denali National Park and Preserve, Alaska, USA \email{jessica\_rykken@nps.gov}}
 }

\maketitle

\end{multicols}

\vspace{-1cm}
\begin{center}
 \parbox[t][][s]{14cm}{\section{Summary}
 We summarize numerous recent changes to the taxonomy of the bumble bee fauna of Alaska since \citet{Pampell2010, Pampell2013}, \citet{Kochetal2012} and \citet{Pampelletal2012, Pampelletal2015}. Nine species are now referred to using different names and two new species were described.

 \begin{enumerate}
\item \citet{Williamsetal2014} resulted in the names \textit{Bombus bohemicus}, \textit{Bombus flavidus}, and \textit{Bombus cryptarum} replacing the previous names of \textit{Bombus ashtoni}, \textit{Bombus fernaldae}, and \textit{Bombus moderatus}, respectively. The former three names replace the latter three for all records in Alaska.

\item \citet{Williamsetal2015} elevated \textit{Bombus natvigi} and \textit{Bombus kirbiellus} from invalid as synonyms under \textit{Bombus hyperboreus} and \textit{Bombus balteatus}, respectively, to valid species status. All former records of \textit{B.\ hyperboreus} in Alaska are now \textit{B.\ natvigi}. All former records of \textit{B.\ balteatus} in Alaska are now \textit{B.\ kirbiellus}.

\item A new species, \textit{Bombus kluanensis}, was described by \citet{Williamsetal2016} from Yukon, Canada and Denali National Park and Preserve, Alaska.

\item Since 2017 we consider \textit{Bombus centralis} to be a doubtful member of the Alaskan fauna with all prior records of this species most likely being \textit{Bombus flavifrons}. 

\item \citet{Martinetetal2019} concluded \textit{Bombus sylvicola} is conspecific with \textit{Bombus lapponicus} and established it as a subspecies, thus all Alaskan \textit{Bombus sylvicola} are now \textit{Bombus lapponicus sylvicola}.

\item A new apparently rare species was described from the Alaskan Arctic: \textit{Bombus interacti} by \citet{Martinetetal2019}.

\item Since December 2019 we consider \textit{Bombus suckleyi} to be a doubtful member of the Alaskan fauna with all prior records of this species most likely being \textit{Bombus bohemicus}. 

\item \citet{Ghisbainetal2020} split \textit{Bombus bifarius} into two species and \textit{B.\ bifarius} does not occur in Alaska. All Alaskan \textit{B.\ bifarius} records should be considered \textit{Bombus vancouverensis}.

\item The Alaskan bumble bee fauna now has 22 confirmed species and 2 doubtful species for a possible total of 24 species.

\end{enumerate}
 }
\end{center}

\vspace{4mm}

\begin{multicols}{2}

\section{Introduction} 

\bibliography{bumble_bee_update}

