\title{Update to the identification guide to female Alaskan bumble bees and a summary of recent changes to the Alaskan bumble bee fauna}

\subtitle{\doi{10.----------}}

\author{by 
 Derek S.\ Sikes\footnote{University of Alaska Museum, Institute of Arctic Biology, University of Alaska Fairbanks, Fairbanks, Alaska, USA \email{dssikes@alaska.edu}} and
 Jessica J.\ Rykken\footnote{Denali National Park and Preserve, Alaska, USA \email{jessica\_rykken@nps.gov}}
 }

\maketitle

\end{multicols}

\vspace{-1cm}
\begin{center}
 \parbox[t][][s]{14cm}{\section{Summary}
 We summarize numerous recent changes to the taxonomy of the bumble bee fauna of Alaska since \citet{Pampell2010, Pampell2013}, \citet{Kochetal2012} and \citet{Pampelletal2012, Pampelletal2015}. Nine species are now referred to using different names and two new species were described.

 \begin{enumerate}
\item Williams et al. (2014) resulted in the names Bombus bohemicus, Bombus flavidus, and Bombus cryptarum replacing the previous names of Bombus ashtoni, Bombus fernaldae, and Bombus moderatus, respectively. The former three names replace the latter three for all records in Alaska.

\item Williams et al. (2015) elevated Bombus natvigi and Bombus kirbiellus from invalid as synonyms under Bombus hyperboreus and Bombus balteatus, respectively, to valid species status. All former records of B. hyperboreus in Alaska are now B. natvigi. All former records of B. balteatus in Alaska are now B. kirbiellus.

\item A new species, Bombus kluanensis, was described by Williams and Cannings (2016) from Yukon, Canada and Denali National Park and Preserve, Alaska.

\item Since 2017 we consider Bombus centralis to be a doubtful member of the Alaskan fauna with all prior records of this species most likely being Bombus flavifrons. 

\item Martinet et al. (2019) concluded Bombus sylvicola is conspecific with Bombus lapponicus and established it as a subspecies, thus all Alaskan Bombus sylvicola are now Bombus lapponicus sylvicola.

\item A new apparently rare species was described from the Alaskan Arctic: Bombus interacti by Martinet et al. (2019).

\item Since December 2019 we consider Bombus suckleyi to be a doubtful member of the Alaskan fauna with all prior records of this species most likely being Bombus bohemicus. 

\item Ghisbain et al. (2020) split Bombus bifarius into two species and B. bifarius doesn't occur in Alaska. All Alaskan B. bifarius records should be considered Bombus vancouverensis.

\item The Alaskan bumble bee fauna now has 22 confirmed species and 2 doubtful species for a possible total of 24 species.

\end{enumerate}
 }
\end{center}

\vspace{4mm}

\begin{multicols}{2}

\section{Introduction} 

\bibliography{bumble_bee_update}

