\title{Misidentifications in science: An example based on \textit{Scathophaga impudicum}  (Diptera: Scathophagidae)}

\subtitle{\doi{10.----------}}

\author{by Derek S.\ Sikes\footnote{University of Alaska Museum, Fairbanks, AK, USA}}

\maketitle

\begin{quotation}
``{\itshape nomina si nescis perit cognitio rerum.}''\\
(``If you know not the names of things, the knowledge of things themselves perishes.'')
\end{quotation}

\hfill---\citet{Coke1628}
\\

I speak on behalf of many entomologists, naturalists, and hopefully most scientists, that organism mis-identifications are an often overlooked but important source of error in scientific practice.

It is not difficult to find examples of misidentification in scientific articles. Such errors often share a number of scientific and editorial failings, notably, an absence of statements on how the identifications were made or who made the identifications, a citation of the source of the taxonomic names, and a statement indicating which public collection the voucher specimen(s) have been, or will be, deposited in (if any museum-quality voucher specimens were prepared at all), which would allow re-examination of the specimens and correction of misidentifications. 

Without the above elements, the use of scientific names approaches the status of unverifiable anecdote at the opposite extreme of rigorous science on which one can base solid conclusions. These lapses are reflective of an all too widespread disregard for whole-organism biology and well-established protocols to maintain high quality science.

In the article by \citet{Croll2005}, published in the peer reviewed journal \textit{Science} and cited 485 times according to Google Scholar, there is a fly species documented from the Aleutian islands of Alaska where this study took place. This fly species, \textit{Scathophaga impudicum}, is mysterious. First, a simple Google search on the name (in quotes, to ensure an exact match to the binomen) finds only hits pointing to the original article and to lecture notes explaining the problems associated with this name. The name with this spelling exists in no online database of scientific names indexed by Google although it can also be found in the related work by the same team, \citet{Maronetal2006}, on which the \textit{Science} article was based. Scientific names are supposed to act a little like passwords---they’re a unique string of letters that should provide access to a wealth of information about that organism. In the case of “\textit{Scathophaga impudicum}” this name is a dead-end.

This name seems to be an alternate spelling of \textit{Scathophaga impudica} (Reiche, 1857) which is a junior synonym of \textit{Scathophaga litorea} (Fallen, 1819) according to \citet{Vockeroth1965} and \citet{Sifner2008}, and thus invalid. \textit{Scathophaga litorea} is a species which occurs on beaches in Europe, Greenland, and eastern North America \citep{Vockeroth1965, Sifner2008, GBIF2020Scathophagalitorea}, but not, as far as any reliable sources indicate, beaches of the Pacific or the Bering Sea. Thus, its use in \citet{Croll2005} and \citet{Maronetal2006} is almost certainly based on a misidentification. 

Additionally, this research found an effect due to fox presence/absence on the marine isotope signatures in this fly species. \textit{Scathophaga} in the Aleutians are predatory shore flies that prey on marine-detritivore shore flies such as \textit{Thoracochaeta}, which are common on beach detritus in the Aleutians \citep[e.g.,][]{Walkeretal2013}. Marine detritus is relatively common on the beaches of all islands regardless of fox presence, which makes the finding of \citet{Croll2005} seem highly implausible. This makes it even more important to understand which fly species (singular or plural) was/were actually sampled. Given the isotopic results it is much more likely to be a fly species that lives inland and is less involved in such a primarily marine-based food web.

Notably missing from the above work is any indication of who did the fly identification (probably not a taxonomist of Diptera), how it was done (morphological or \acr{DNA} or ?), where the voucher specimens are deposited, or even if any voucher specimens were saved and deposited, and from what publication the name \textit{Scathophaga impudicum} came. Ideally, taxonomic names should come from a recent taxonomic revision, catalog, or taxonomic name server indexed by Google.

The loss in training and funding of traditional taxonomic skills is often justified in part on the incorrect notion that identification of organisms is easy enough to accomplish that no special mention of the process is required. Those taxonomists who curate names and specimens, and can provide reliable identifications, are professional biologists and should be credited with their contribution to a study.

 As the above example illustrates, proper identification of organisms is not trivial and when errors arise the results can range from embarrassment to irreproducible science. If one’s science is to be rigorous, readers must be given enough information to judge the quality of the identifications and access to voucher specimens to verify any identifications that might be questionable. This is, of course, even more critical when cryptic species are discovered after a study has been completed. Without voucher specimens it may be impossible to know which of two or more sympatric cryptic species was actually studied.

As \citet{Packeretal2018} pointed out, who performed an excellent review of these problems in the entomological literature, there is vast room for improvement in proper documentation supporting scientific identifications. We all must do better.


\section{Acknowledgements}

Thanks to the author of the blog post which corrected my misunderstanding of who wrote the Latin quote at the start of this paper---not Linnaeus but Edward Coke (\url{http://languagehat.com/three-years-of-languagehat/}).

%\bibliography{misidentification}

\begin{thebibliography}{8}
\expandafter\ifx\csname natexlab\endcsname\relax\def\natexlab#1{#1}\fi
\expandafter\ifx\csname url\endcsname\relax
  \def\url#1{{\tt #1}}\fi
\expandafter\ifx\csname urlprefix\endcsname\relax\def\urlprefix{{\small URL}
  }\fi

\bibitem[{Coke(1628)}]{Coke1628}
Coke, E.
\newblock 1628.
\newblock The First Part of the Institutes of the Lawes of England, or, A
  Commentary upon Littleton, not the Name of the Author Only, but of the Law It
  Selfe.
\newblock Printed for the Societie of Stationers, London.
\newblock
  \urlprefix\url{https://archive.org/details/firstpartofinsti011628coke}.

\bibitem[{Croll et~al.(2005)Croll, Maron, Estes, Danner, and Byrd}]{Croll2005}
Croll, D.~A., J.~L. Maron, J.~A. Estes, E.~M. Danner, and G.~V. Byrd.
\newblock 2005.
\newblock Introduced predators transform subarctic islands from grassland to
  tundra.
\newblock Science {\bfseries 307}:1959--1961.
\newblock \doi{10.1126/science.1108485},
  \urlprefix\url{https://science.sciencemag.org/content/307/5717/1959}.

\bibitem[{{GBIF Secretariat}(2020)}]{GBIF2020Scathophagalitorea}
{GBIF Secretariat}.
\newblock 2020.
\newblock \textit{Scathophaga litorea} (Fallen, 1819) \textit{in} GBIF Backbone
  Taxonomy. Checklist dataset accessed via GBIF.org on 2020-04-15.
\newblock \doi{10.15468/39omei},
  \urlprefix\url{https://www.gbif.org/species/1556255}.

\bibitem[{Maron et~al.(2006)Maron, Estes, Croll, Danner, Elmendorf, and
  Buckelew}]{Maronetal2006}
Maron, J.~L., J.~A. Estes, D.~A. Croll, E.~M. Danner, S.~C. Elmendorf, and
  S.~L. Buckelew.
\newblock 2006.
\newblock An introduced predator alters {Aleutian} island plant communities by
  thwarting nutrient subsidies.
\newblock Ecological Monographs {\bfseries 76}:3--24.
\newblock \doi{10.1890/05-0496}.

\bibitem[{Packer et~al.(2018)Packer, Monckton, Onuferko, and
  Ferrari}]{Packeretal2018}
Packer, L., S.~K. Monckton, T.~M. Onuferko, and R.~R. Ferrari.
\newblock 2018.
\newblock Validating taxonomic identifications in entomological research.
\newblock Insect Conservation and Diversity {\bfseries 11}:1--12.
\newblock \doi{10.1111/icad.12284}.

\bibitem[{Vockeroth(1965)}]{Vockeroth1965}
Vockeroth, J.~R.
\newblock 1965.
\newblock Subfamily {Scatophaginae}.
\newblock Pp.\ 826--842 {\em in\/} A.~Stone, C.~W. Sabrosky, W.~W. Wirth, R.~H.
  Foote, and J.~R. Coulson, editors. A catalogue of the {Diptera} of {America}
  north of {Mexico}. Agricultural Research Service, United States Department of
  Agriculture, Washington, D.C.
\newblock
  \urlprefix\url{https://naldc-legacy.nal.usda.gov/naldc/catalog.xhtml?id=CAT87208336}.

\bibitem[{\v{S}ifner(2008)}]{Sifner2008}
\v{S}ifner, F.
\newblock 2008.
\newblock A catalogue of the Scathophagidae (Diptera) of the Palaearctic
  region, with notes on their taxonomy and faunistics.
\newblock Acta Entomologica Musei Nationalis Pragae {\bfseries 48}:111--196.
\newblock \urlprefix\url{http://www.aemnp.eu/pdf/48_1/48_1_111.pdf}.

\bibitem[{Walker et~al.(2013)Walker, Sikes, Degange, Jewett, Michaelson,
  Talbot, Talbot, Wang, and Williams}]{Walkeretal2013}
Walker, L.~R., D.~S. Sikes, A.~R. Degange, S.~C. Jewett, G.~Michaelson, S.~L.
  Talbot, S.~S. Talbot, B.~Wang, and J.~C. Williams.
\newblock 2013.
\newblock Biological legacies: Direct early ecosystem recovery and food web
  reorganization after a volcanic eruption in Alaska.
\newblock Écoscience {\bfseries 20}:240--251.
\newblock \doi{10.2980/20-3-3603}.

\end{thebibliography}
